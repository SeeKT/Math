\documentclass{jsreport}
\usepackage{graphicx, url, algorithm, algorithmic, float, booktabs, listings, jlisting, color, pdfpages, amsmath, amssymb, latexsym, mathtools}
\lstset{
  basicstyle={\ttfamily},
  identifierstyle={\small},
  commentstyle={\smallitshape},
  keywordstyle={\small\bfseries},
  ndkeywordstyle={\small},
  stringstyle={\small\ttfamily},
  frame={tb},
  breaklines=true,
  columns=[l]{fullflexible},
  numbers=left,
  xrightmargin=0zw,
  xleftmargin=3zw,
  numberstyle={\scriptsize},
  stepnumber=1,
  numbersep=1zw,
  lineskip=-0.5ex
}
\usepackage[table,xcdraw]{xcolor}
\newtheorem{theo}{Theorem}[section]
\newtheorem{defi}{Definition}[section]
\newtheorem{lemm}{Lemma}[section]
\newcommand{\red}[1]{\textcolor{red}{#1}}
\newcommand{\blue}[1]{\textcolor{blue}{#1}}
\newcommand{\green}[1]{\textcolor{green}{#1}}
\renewcommand{\figurename}{図}
\renewcommand{\tablename}{表}
\renewcommand{\baselinestretch}{1.1}
\usepackage[hang,small,bf]{caption}
\usepackage[subrefformat=parens]{subcaption}
\usepackage{mathrsfs}

\begin{document}
\chapter{2変量 (多変量) 確率ベクトルの分布}
\section{$n$次元確率ベクトルの同時分布}
一般に,$n$個の1次元確率変数$X_1, X_2, \ldots, X_n$に対して,
$\bm{X} = (X_1, X_2, \ldots, X_n)$を$n$次元確率ベクトル (random vector, r.vec.),
$P(a_1 \leq X_1 \leq b_1, a_2 \leq X_2 \leq b_2, \cdots, a_n \leq X_n \leq b_n), \, a_i \leq b_i \, (i = 1, 2, \ldots, n)$を$\bm{X}$の同時確率分布という.
$F_{\bm{X}}(x_1, \cdots x_n) = P(X_1 \leq x_1, \cdots, X_n \leq x_n)$を$\bm{X}$の同時分布関数という.以降,主に連続型について述べる.

\begin{screen}
  \begin{defi}[同時確率密度関数]
    $\bm{X} = (X_1, \ldots, X_n)$を$n$次元確率ベクトルとする.
    \begin{equation}
      P(a_1 \leq X_1 \leq b_1, \cdots, a_n \leq X_n \leq b_n) = \int_{a_1}^{b_1} \cdots \int_{a_n}^{b_n} f_{\bm{X}}(x_1, \cdots, x_n) dx_1 \cdots dx_n \nonumber
    \end{equation}
    で与えられ,次を満たす$f_{\bm{X}}(x_1, \cdots, x_n)$を$\bm{X}$の(n次元)同時確率密度関数(joint p.d.f.)という.
    \begin{enumerate}
      \item $f_{\bm{X}}(x_1, \cdots, x_n) \geq 0$.
      \item $\int_{-\infty}^{\infty} \cdots \int_{-\infty}^{\infty} f_{\bm{X}}(x_1, \cdots, x_n) dx_1 \cdots dx_n = 1$.
    \end{enumerate}
  \end{defi}
\end{screen}

同時確率密度関数は,
\begin{equation}
  f_{\bm{X}}(x_1, \cdots, x_n) = \frac{\partial^n}{\partial x_1 \cdots \partial x_n} F_{\bm{X}}(x_1, \cdots, x_n) \nonumber
\end{equation}
および
\begin{equation}
  F_{\bm{X}}(x_1, \cdots, x_n) = \int_{-\infty}^{x_1} \cdots \int_{-\infty}^{x_n} f_{\bm{X}}(y_1, \cdots, y_n) dy_1 \cdots dy_n \nonumber
\end{equation}
を満たす.

各成分だけに注目した分布関数
\begin{align}
  F_i(x_i) &= F_{\bm{X}}(\infty, \cdots, \infty, x_i, \infty, \cdots, \infty) \nonumber \\
   &= P(X_1 \in \mathbb{R}, \cdots, X_{i - 1} \in \mathbb{R}, X_i \leq x_i, X_{i + 1} \in \mathbb{R}, \cdots, X_n \in \mathbb{R}) \nonumber \\
   &= P(X_i \leq x_i) \nonumber
\end{align}
を$X_i$の周辺分布関数という.その分布関数を
\begin{equation}
  f_{X_i}(x_i) = \frac{\partial F_i(x_i)}{\partial x_i} = \int_{-\infty}^{\infty} \cdots \int_{-\infty}^{\infty} f(x_1, \cdots, x_n) dx_1 \cdots dx_{i - 1} dx_{i + 1} \cdots dx_n \nonumber
\end{equation}
を周辺確率密度関数という.周辺確率密度関数は,
\begin{equation}
  \int_{-\infty}^{x_i} f_{X_i}(x) dx = F_i(x_i) \nonumber
\end{equation}
を満たす.


\end{document}
