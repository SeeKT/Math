\documentclass{jsreport}
\usepackage{graphicx, url, algorithm, algorithmic, float, booktabs, listings, jlisting, color, pdfpages, amsmath, amssymb, latexsym, mathtools}
\lstset{
  basicstyle={\ttfamily},
  identifierstyle={\small},
  commentstyle={\smallitshape},
  keywordstyle={\small\bfseries},
  ndkeywordstyle={\small},
  stringstyle={\small\ttfamily},
  frame={tb},
  breaklines=true,
  columns=[l]{fullflexible},
  numbers=left,
  xrightmargin=0zw,
  xleftmargin=3zw,
  numberstyle={\scriptsize},
  stepnumber=1,
  numbersep=1zw,
  lineskip=-0.5ex
}
\usepackage[table,xcdraw]{xcolor}
\newtheorem{theo}{Theorem}[section]
\newtheorem{defi}{Definition}[section]
\newtheorem{lemm}{Lemma}[section]
\newcommand{\red}[1]{\textcolor{red}{#1}}
\newcommand{\blue}[1]{\textcolor{blue}{#1}}
\newcommand{\green}[1]{\textcolor{green}{#1}}
\renewcommand{\figurename}{図}
\renewcommand{\tablename}{表}
\renewcommand{\baselinestretch}{1.1}
\usepackage[hang,small,bf]{caption}
\usepackage[subrefformat=parens]{subcaption}
\usepackage{mathrsfs}

\begin{document}
\chapter{確率分布の代表的モデル}
\section{離散分布モデル}
ここでは,離散一様分布,二項分布,ポアソン分布について,その分布の特性値を求める.
\subsection{離散一様分布 $DU(n)$}
確率変数$X$が値$1, \ldots, n$を等確率$1/n$でとるとき,その確率関数は
\begin{equation}
  f(x) = P(X = x) = \frac{1}{n}, \; \; x = 1, \ldots, n \nonumber
\end{equation}
である.このような分布を離散一様分布 (discrete uniform distribution) といい,記号$DU(n)$で表す.

$DU(n)$について,平均,分散はそれぞれ,
\begin{align}
  &E[X] = \sum_{x = 1}^n x \frac{1}{n} = \frac{n + 1}{2} \nonumber \\
  &V[X] = \sum_{x = 1}^n x^2 \frac{1}{n} - (E[X])^2 = \frac{(n + 1)(2n + 1)}{6} - \frac{(n + 1)^2}{4} = \frac{n^2 - 1}{12} \nonumber
\end{align}
で与えられる.

\subsection{二項分布 $B_N(n, p)$}
成功の確率が$p \, (0 < p < 1)$,失敗の確率が$1 - p$の試行をベルヌーイ試行 (Bernoulli trial) という.その試行の成功を$1$,失敗を$0$とするとき,2値 (bivariate) r.v. $\varepsilon$が得られる:
\begin{equation}
  \varepsilon = \begin{cases}
    1 & (確率p), \\
    0 & (確率1 - p).
\end{cases}\nonumber
\end{equation}
この分布をベルヌーイ分布といい,記号$Ber(p)$で表す.確率関数は,
\begin{equation}
  f(\varepsilon | p) = p^{\varepsilon} (1 - p)^{\varepsilon}, \; \varepsilon = 0, 1 \nonumber
\end{equation}
となる.特に,成功の確率と失敗の確率の比$p/(1-p)$をオッズ (odds) という.

ベルヌーイ分布の平均と分散は,
\begin{align}
  &E[\varepsilon] = 1 \times p + 0 \times (1 - p) = p, \nonumber \\
  &V[\varepsilon] = E[\varepsilon^2] - (E[\varepsilon])^2 = p - p^2 = p(1 - p). \nonumber
\end{align}

次に,$n$回の独立なベルヌーイ試行を$\varepsilon_1, \varepsilon_2, \ldots, \varepsilon_n$とし,そのときの成功の回数を$X = \varepsilon_1 + \varepsilon_2 + \cdots + \varepsilon_n$とおくと,$X$は離散値
$0, 1, \ldots, n$をとるr.v.で,その確率関数は次のようになる:
\begin{equation}
  f(x|p) = \binom{n}{x} p^x (1 - p)^{n - x}, \; \; x = 0, 1, \ldots, n. \nonumber
\end{equation}

二項定理より,
\begin{equation}
  \sum_{x = 0}^n f(x|p) = \sum_{x = 0}^n \binom{n}{x} p^x (1 - p)^{n - x} = (p + (1 - p))^n = 1. \nonumber
\end{equation}
この分布を二項分布 (binomial distribution) といい,記号$B_N(n, p)$で表す.ただし,$B_N(n, p) = Ber(p)$である.

p.g.f.は,
\begin{align}
  P(t) &= E[t^X] = \sum_{x = 0}^{\infty} t^x \binom{n}{x} p^x (1 - p)^{n - x} \nonumber \\
  &= \sum_{x = 0}^{\infty} \binom{n}{x} (tp)^x (1 - p)^{n - x} = (pt + 1 - p)^n \nonumber
\end{align}
である.この微分を計算することで,$k$次階乗モーメントが計算できる:
\begin{align}
  &P^{\prime}(t) = np(pt + 1 - p)^{n - 1}, \; \; \therefore E[X] = P^{\prime}(1) = np. \nonumber \\
  &P^{\prime \prime}(t) = n(n - 1)p^2(pt + 1 - p)^{n - 2}, \; \; \therefore E[X(X - 1)] = P^{\prime \prime}(1) = n(n - 1)p^2. \nonumber
\end{align}
これから分散も計算でき,
\begin{equation}
  V[X] = np(1 - p) \nonumber
\end{equation}
が得られる.

\subsection{ポアソン分布 $P_O(\lambda)$}
「まれな現象の大量観測」によって発生する事象の個数は,ポアソン分布 (Poisson distribution) に従う.例えば,1台の自動車が1日に交通事故を起こす確率は小さいが,自動車の台数は多いので,1日の交通事故の件数はポアソン分布に従うことが知られている.ポアソン分布は記号$P_O(\lambda)$で表し,その確率関数は,
\begin{align}
  f(x | \lambda) &= \frac{\lambda^x}{x!} e^{-\lambda}, \; \; x = 0, 1, 2, \ldots, \nonumber \\
  \Theta &= \{\lambda : 0 < \lambda < \infty\} \nonumber
\end{align}
で与えられる.母数$\lambda$のことを強度 (intensity) という.
指数関数のべき級数展開により,
\begin{equation}
  \sum_{x = 0}^{\infty} f(x | \lambda) = e^{-\lambda} \sum_{x = 0}^{\infty} \frac{\lambda^x}{x!} = e^{-\lambda}e^{\lambda} = 1
\end{equation}
が得られる.

ポアソン分布の確率母関数は,
\begin{equation}
  P(t) = E[t^X] = e^{-\lambda} \sum_{x = 0}^{\infty} \frac{t^x \lambda^x}{x!} = e^{-\lambda}e^{t\lambda} = e^{\lambda(t - 1)}. \nonumber
\end{equation}
であり,
\begin{align}
  &P^{\prime}(t) = \lambda e^{\lambda(t - 1)} \; \; \therefore E[X] = P^{\prime}(1) = \lambda \nonumber \\
  &P^{\prime \prime}(t) = \lambda^2 e^{\lambda(t - 1)} \; \; \therefore E[X(X - 1)] = P^{\prime \prime}(1) = \lambda^2 \nonumber
\end{align}
から分散は$V[X] = \lambda$と求められる.つまり,ポアソン分布の平均と分散は等しく強度$\lambda$である.
成功確率が小さいが,試行回数が大きいとき,二項分布に対する近似分布として,ポアソン分布が導出される.

\section{連続分布モデル}
ここでは,一様分布および正規分布について,その分布の特性値を求める.
\subsection{一様分布 $U(\alpha, \beta)$}
確率変数$X$が区間$(\alpha, \beta)$上の値を等確率でとるとき,その密度関数は,
\begin{align}
  f(x) &= \begin{cases}
    \frac{1}{\beta - \alpha} & (\alpha < x < \beta) \\
    0 & (\mbox{otherwise}).
\end{cases}\nonumber \\
\Theta &= \{\bm{\theta} = (\alpha, \beta) : -\infty < \alpha < \beta < \infty\} \nonumber
\end{align}
と記述される.このような分布を一様分布 (uniform distribution) といい,記号$U(\alpha, \beta)$で表す.

一様分布の平均および分散は,
\begin{align}
  &E[X] = \frac{\alpha + \beta}{2} \nonumber \\
  &V[X] = \frac{(\beta - \alpha)^2}{12} \nonumber
\end{align}
と求められる.

\subsection{正規分布}
正規分布 (normal distribution) / ガウス分布 (Gaussian distribution) は,次の密度関数を持つ.
\begin{align}
  f(x | \bm{\theta}) &= \frac{1}{\sigma \sqrt{2\pi}} \exp\left\{- \frac{(x - \mu)^2}{2 \sigma^2} \right\}, \; \; -\infty < x < \infty, \nonumber \\
  \Theta &= \{\bm{\theta} = (\mu, \sigma^2) : -\infty < \mu < \infty, \, 0 < \sigma^2 < \infty\}. \nonumber
\end{align}
正規分布は母数$\mu, \sigma^2$にのみ依存するので,記号$N(\mu, \sigma^2)$で表す.
この母数$\mu, \sigma^2$はそれぞれ平均と分散である.特に,$\mu = 0, \sigma^2 = 1$のとき,$N(0,1)$を標準正規分布 (standard normal distribution) という.標準正規分布の密度関数は
\begin{equation}
  \varphi(z) = \frac{1}{\sqrt{2\pi}} \exp\left(- \frac{z^2}{2}\right) \nonumber
\end{equation}
という記号で表す.一般の正規分布$N(\mu, \sigma^2)$の密度関数は,
\begin{equation}
  f(x | \bm{\theta}) = \frac{1}{\sigma} \varphi\left(\frac{x - \mu}{\sigma}\right) \nonumber
\end{equation}
のように,$\varphi(z)$の線形変換として記述できる.

\begin{description}
  \item[標準化]
  r.v. $X$が平均$\mu$,分散$\sigma^2$を持つとする.このとき,
  \begin{equation}
    Z \coloneqq \frac{X - \mu}{\sigma} \nonumber
  \end{equation}
  とすると,$E[Z] = 0, \, V[Z] = 1$となる.この変換を標準化という.
\end{description}

標準化を用いると,
\begin{equation}
  X \sim N(\mu, \sigma^2) \Longrightarrow Z = \frac{X - \mu}{\sigma} \sim N(0, 1) \nonumber
\end{equation}
とすることができる.

つまり,確率変数$X \sim N(\mu, \sigma^2)$の性質は標準化によって,$Z \sim N(0, 1)$の性質に変換することができる.逆に,$Z \sim N(0, 1)$から$X \sim N(\mu, \sigma^2)$を構成することができる.

標準正規分布$N(0, 1)$の積率母関数は,
\begin{equation}
  M_z(t) = E[e^{tZ}] = \int_{-\infty}^{\infty} e^{tz}\varphi(z)dz = e^{\frac{t^2}{2}} \nonumber
\end{equation}
であり,正規分布$N(\mu, \sigma^2)$の積率母関数は,
\begin{equation}
  M_x(t) = E[e^{tX}] = E[e^{t(\mu + \sigma Z)}] = e^{\mu t} E[e^{(t\sigma})Z] = e^{\mu t + \frac{\sigma^2 t^2}{2}} \nonumber
\end{equation}
である.


\end{document}
