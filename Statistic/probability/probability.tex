\documentclass{jsreport}
\usepackage{graphicx, url, algorithm, algorithmic, float, booktabs, listings, color, pdfpages, amsmath, amssymb, latexsym, mathtools, ascmac, amsfonts, amsthm}
\lstset{
  basicstyle={\ttfamily},
  identifierstyle={\small},
  commentstyle={\smallitshape},
  keywordstyle={\small\bfseries},
  ndkeywordstyle={\small},
  stringstyle={\small\ttfamily},
  frame={tb},
  breaklines=true,
  columns=[l]{fullflexible},
  numbers=left,
  xrightmargin=0zw,
  xleftmargin=3zw,
  numberstyle={\scriptsize},
  stepnumber=1,
  numbersep=1zw,
  lineskip=-0.5ex
}
\usepackage[table,xcdraw]{xcolor}
\newtheorem{theo}{定理}[chapter]
\newtheorem{defi}{定義}[chapter]
\newtheorem{lemm}{補題}[chapter]
\newtheorem{prop}{命題}[chapter]
\newtheorem{coro}{系}[chapter]
\newcommand{\red}[1]{\textcolor{red}{#1}}
\newcommand{\blue}[1]{\textcolor{blue}{#1}}
\newcommand{\green}[1]{\textcolor{green}{#1}}
\renewcommand{\baselinestretch}{1.1}
\usepackage{mathrsfs}
\usepackage{bm}
\def\qed{\hfill $\Box$}
\usepackage{tikz}
\usetikzlibrary{intersections, calc, arrows}
\renewcommand\proofname{\bf 証明}

\begin{document}

\chapter{事象と確率}
\section{集合と事象}
\begin{itemize}
  \item 試行 (trials)
  \begin{itemize}
    \item 実験や観測,調査の総称.
  \end{itemize}
  \item 全事象 (total event)
  \begin{itemize}
    \item 試行を行ったときに起こりうる全ての結果の集まり
  \end{itemize}
  \item 母集団 (population)
  \begin{itemize}
    \item 統計学では,全事象のことを試行の対象となる集団とみなす.
  \end{itemize}
  \item 事象 (event)
  \begin{itemize}
    \item ある性質を満たす結果の集まり
  \end{itemize}
\end{itemize}

$\Omega$をある集合 (全集合)とし,$\emptyset$を空集合 (空事象) とする.可測集合の定義を以下で与える.
\begin{screen}
  \begin{defi}
    $\Omega$の部分集合を要素とする集合族$\mathscr{A}$が次の3条件を満たすとき,$\sigma$-加法族という.
    \begin{enumerate}
      \item $\Omega \in \mathscr{A}, \; \emptyset \in \mathscr{A}$.
      \item $A \in \mathscr{A}$ならば$A^{\mathrm{C}} \in \mathscr{A}$.
      \item $A_n \in \mathscr{A} \, (n = 1, 2, \ldots)$ならば,$\bigcup_{n = 1}^{\infty} A_n \in \mathscr{A}$.
    \end{enumerate}
  \end{defi}
\end{screen}
また,$(\Omega, \mathscr{A})$を可測空間,$\mathscr{A}$の要素を可測集合という\footnote{テキストでは,$\mathscr{A}$($\sigma$-加法族)を事象族,$\mathscr{A}$の要素を事象と呼んでいる.}.

\section{確率と確率空間}
\begin{screen}
  \begin{defi}\label{def:pr_measure}
    事象族$\mathscr{A}$上で定義された関数$P$が次の3条件を満たすとき,確率 (または確率測度 (probability measure) ) という.
    \begin{description}
      \item[(P1)] 任意の事象$A \in \mathscr{A}$に対して,$0 \leq P(A) \leq 1$.
      \item[(P2)] $P(\emptyset) = 0, \; P(\Omega) = 1$.
      \item[(P3)] $A_n \in \mathscr{A} \, (n = 1, 2, \ldots)$が互いに素 ($A_m \cap A_n = \emptyset \, (m \leq n)$)のとき\footnote{$\sigma-加法性$という.},
      \begin{equation}
        P(\bigcup_{n = 1}^{\infty} A_n) = \sum_{n = 1}^{\infty} P(A_n). \; \; \nonumber
      \end{equation}
    \end{description}
  \end{defi}
\end{screen}
このとき,$(\Omega, \mathscr{A}, P)$を確率空間 (probability space)と呼ぶ.
(P1)-(P3)から確率測度に関する基本的性質が導かれる:
\begin{enumerate}
  \item $P(A^{\mathrm{C}}) = 1 - P(A)$
  \item $A \subset B \Longrightarrow P(A) \leq P(B)$
  \item $P(A \cup B) = P(A) + P(B) - P(A \cap B)$
  \item $A_n \in \mathscr{A}, \, A_n \supset A_{n + 1} \, (n = 1, 2, \ldots) \Longrightarrow P(\bigcap_{n = 1}^{\infty} A_n) = \lim_{n \to \infty} P(A_n)$
  \item $A_n \in \mathscr{A}, \, A_n \subset A_{n + 1} \, (n = 1, 2, \ldots) \Longrightarrow P(\bigcup_{n = 1}^{\infty} A_n) = \lim_{n \to \infty} P(A_n)$
\end{enumerate}

\section{事象の独立性と従属性}
\begin{screen}
  \begin{defi}[独立性]
    2つの事象$A, \, B$が独立 (independent) であるとは,
    \begin{equation}
      P(A \cap B) = P(A) P(B) \nonumber
    \end{equation}
    が成り立つことをいう.独立でないとき,従属 (dependent) であるという.
  \end{defi}
\end{screen}

\begin{screen}
  \begin{defi}[条件付き確率]\label{def:cond_pr}
    2つの事象$A, \, B$があって,$P(A) > 0$のとき,
    \begin{equation}
      P(B|A) \stackrel{\mathrm{def}}{\coloneqq} \frac{P(A \cap B)}{P(A)} \nonumber
    \end{equation}
    を,$A$が与えられたときの$B$の条件付き確率という.
  \end{defi}
\end{screen}
条件付き確率について,以下の性質が成り立つ.
\begin{enumerate}
  \item $P(A \cap B) = P(A) P(B|A)$
  \item $A, \, B$が独立のとき,$P(B|A) = P(B), \, P(A|B) = P(A)$
\end{enumerate}

\begin{screen}
  \begin{theo}[全確率の法則]\label{theo:total_pr}
    $A_1, A_2 \ldots, A_n$を互いに素な事象で,$\Omega = \bigcup_{i = 1}^n A_i$とする.
    $P(A_i) > 0 \, (i = 1, 2, \ldots, n)$のとき,任意の事象$B$に対して,
    \begin{equation}\label{eq:total_pr}
      P(B) = \sum_{i = 1}^n P(A_i) P(B | A_i)
    \end{equation}
    が成立する.
  \end{theo}
\end{screen}

\begin{proof}[定理\ref{theo:total_pr}の証明]
  \begin{equation}
    B = B \cap \Omega = B \cap \left(\bigcup_{i = 1}^n A_i\right) = \bigcup_{i = 1}^n (B \cap A_i) \nonumber
  \end{equation}
  および
  \begin{equation}
    (B \cap A_i) \cap (B \cap A_j) = \emptyset \; (i \neq j) \nonumber
  \end{equation}
  より,定義\ref{def:pr_measure}から
  \begin{equation}
    P(B) = P\left(\bigcup_{i = 1}^n (B \cap A_i)\right) = \sum_{i = 1}^n P(B \cap A_i) \nonumber
  \end{equation}
  が成立し,定義\ref{def:cond_pr}より,(\ref{eq:total_pr})式が成立する.\qed
\end{proof}

事象$B$が起こる前と後のことをそれぞれ事前と事後と呼ぶ.$A_i$に対して$P(A_i)$は$B$が起こる前に与えられている確率であるから事前確率 (prior probability) といい,$P(A_i | B)$は$B$が起こった後で与えられる確率であるから事後確率 (posterior probability) という.次のベイズの定理が成立する\footnote{ベイズの定理は,事前確率$P(A_i),\, P(B|A_i) \; (i = 1, 2, \ldots, n)$が求まれば,事後確率$P(A_i|B)$が導出できることを述べている.}.
\begin{screen}
  \begin{theo}[ベイズの定理]\label{theo:bayes}
    $A_1, A_2, \ldots, A_n$を互いに素な事象で,$\Omega = \bigcup_{i = 1}^n A_i$とする.
    $P(A_i) > 0 \; (i = 1,2, \ldots, n)$のとき,任意の事象$B$に対して
    \begin{equation}\label{eq:bayes}
      P(A_i | B) = \frac{P(A_i) P(B | A_i)}{\sum_{j = 1}^n P(A_j) P(B | A_j)} \; \; (i = 1, 2, \ldots, n)
    \end{equation}
    が成立する.
  \end{theo}
\end{screen}

\begin{proof}[定理\ref{theo:bayes}の証明]
  定義\ref{def:cond_pr}より,
  \begin{equation}
    P(A_i|B) = \frac{P(A_i \cap B)}{P(B)}. \nonumber
  \end{equation}
  条件付き確率の性質より,
  \begin{equation}
    P(A_i \cap B) = P(A_i) P(B | A_i). \nonumber
  \end{equation}
  定理\ref{theo:total_pr}より,
  \begin{equation}
    P(B) = \sum_{i = 1}^n P(A_i) P(B | A_i)
  \end{equation}
  であるから,(\ref{eq:bayes})式が成立する.\qed
\end{proof}



\end{document}
