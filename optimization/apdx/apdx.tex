\documentclass[dvipdfmx]{jsreport}
\usepackage{graphicx, url, algorithm, algorithmic, float, booktabs, listings, color, pdfpages, amsmath, amssymb, latexsym, mathtools, ascmac, amsfonts, amsthm}
\lstset{
  basicstyle={\ttfamily},
  identifierstyle={\small},
  commentstyle={\smallitshape},
  keywordstyle={\small\bfseries},
  ndkeywordstyle={\small},
  stringstyle={\small\ttfamily},
  frame={tb},
  breaklines=true,
  columns=[l]{fullflexible},
  numbers=left,
  xrightmargin=0zw,
  xleftmargin=3zw,
  numberstyle={\scriptsize},
  stepnumber=1,
  numbersep=1zw,
  lineskip=-0.5ex
}
\usepackage[table,xcdraw]{xcolor}
\newtheorem{theo}{定理}[chapter]
\newtheorem{defi}{定義}[chapter]
\newtheorem{lemm}{補題}[chapter]
\newtheorem{prop}{命題}[chapter]
\newtheorem{coro}{系}[chapter]
\newcommand{\red}[1]{\textcolor{red}{#1}}
\newcommand{\blue}[1]{\textcolor{blue}{#1}}
\newcommand{\green}[1]{\textcolor{green}{#1}}
\renewcommand{\baselinestretch}{1.1}
\usepackage{mathrsfs}
\usepackage{bm}
\def\qed{\hfill $\Box$}
\usepackage{tikz}
\usetikzlibrary{intersections, calc, arrows}
\renewcommand\proofname{\bf 証明}

\begin{document}
\chapter{証明}
この章には,補題や定理の証明を記載する.

\begin{proof}[補題\ref{lemm:convex}の証明]
  項が2つのとき,つまり,$X_1, X_2 \subseteq \mathbb{R}^n$が凸集合ならば,$X_1 \cap X_2$も凸集合である(*)
  を証明する.$\forall \bm{x}^1, \bm{x}^2 \in X_1 \cap X_2$に対し,$\bm{x}^1$と$\bm{x}^2$を結ぶ線分は,$\bm{x}^1 \in X_1, \bm{x}^2 \in X_1$より$X_1$に含まれる($\because$ $X_1$は凸集合).同様に$X_2$にも含まれる.
  よって,$\bm{x}^1$と$\bm{x}^2$を結ぶ線分は,$X_1 \cap X_2$に含まれる.つまり,$X_1 \cap X_2$は凸集合である.補題\ref{lemm:convex}は,(*)を繰り返し用いることで示される.\qed
\end{proof}

補題\ref{lemm:hyperplane1}の証明には,定理\ref{theo:projection},補題\ref{lemm:proj}を用いる.

\begin{theo}[射影定理]\label{theo:projection}
  $X$をHilbert空間\footnote{距離(ノルム)を持つ集合をノルム空間という.内積を持つ線形空間を内積空間という.ノルム空間$X$内の任意のコーシー列が収束するとき,$X$は完備であるといい,完備性を持つノルム空間$X$をBanach空間という.また,内積空間$X$上の点$\bm{u} \in X$に対し,$\|\bm{u}\| = \sqrt{(\bm{u}, \bm{u})}$を内積から誘導されるノルムと呼ぶ.内積から誘導されるノルム空間$X$がBanach空間であるとき,$X$をHilbert空間という.実数空間$\mathbb{R}^n$は完備性を持つ.}とし,$L \subset X$を閉部分空間とする.このとき,
  \begin{equation}
    \bm{u} \in X \, (\mathrm{given}) \Rightarrow \exists! \bm{v} \in L \; s.t. \; (\bm{u} - \bm{v}, \bm{w}) = 0, \, \forall \bm{w} \in L \nonumber
  \end{equation}
  が成立する(射影が一意に存在する).
\end{theo}

\begin{lemm}\label{lemm:proj}
    定理\ref{theo:projection}の$X$が$X = \mathbb{R}^n$であり,$L \subseteq \mathbb{R}^n$が空でない閉凸集合とする.$\bm{u} \in \mathbb{R}^n$の$L$への射影を$\bm{v}$とする.このとき,$\forall \bm{w} \in L$に対して,
  \begin{equation}\label{eq:proj}
    (\bm{u} - \bm{v})^{\mathrm{T}} (\bm{w} - \bm{v}) \leq 0
  \end{equation}
  が成立する.
\end{lemm}

\begin{proof}[定理\ref{theo:projection}の証明]

\end{proof}

\begin{proof}[補題\ref{lemm:proj}の証明]
  $\bm{u}$の$L$への射影$\bm{v}$と任意の点$\bm{w} \in L$を結ぶ線分上の点
  $\bm{x}_{\lambda} = (1 - \lambda)\bm{v} + \lambda \bm{w}, \, 0 < \lambda < 1$
  を考える.$L$は凸より,$\bm{x}_{\lambda}$も$L$に含まれる.
  $\bm{v}$の定義より,
  \begin{equation}\label{eq:prj_min}
    \|\bm{v} - \bm{u}\|^2 \leq \|((1 - \lambda)\bm{v} + \lambda \bm{w}) - \bm{u} \|^2
  \end{equation}
  (\ref{eq:prj_min})式を整理すると,
  \begin{align}
    (\bm{v}- \bm{u}, \bm{v}- \bm{u}) &\leq \|(\bm{v} - \bm{u}) + \lambda(\bm{w} - \bm{v}) \|^2 \nonumber \\
    &\leq (\bm{v}- \bm{u}, \bm{v}- \bm{u}) + 2\lambda(\bm{v} - \bm{u})(\bm{w} - \bm{v}) + \lambda^2(\bm{w} - \bm{v}, \bm{w} - \bm{v}) \nonumber
  \end{align}
  いま,$\lambda \neq 0, \lambda > 0$であり,$L \subseteq \mathbb{R}^n$であるから,
  \begin{equation}
    2(\bm{u} - \bm{v})(\bm{w} - \bm{v}) \leq \lambda \|\bm{w} - \bm{v}\|^2 \nonumber
  \end{equation}
  よって,$\lambda \rightarrow 0$とすると,(\ref{eq:proj})式を得る.\qed
\end{proof}

\begin{proof}[補題\ref{lemm:hyperplane1}の証明]
  $\bm{y} \in \mathbb{R}^n$の$X$への射影を$\bar{\bm{y}}$とする.定理\ref{theo:projection}より,$\bar{\bm{y}}$は$\bm{y}$に対して一意に存在する.$\bm{y} \notin X$より,$\bm{y} \neq \bar{\bm{y}}$である.
  よって,
  \begin{equation}\label{eq:norm}
    \|\bar{\bm{y}} - \bm{y}\|^2 = (\bar{\bm{y}} - \bm{y})^{\mathrm{T}}(\bar{\bm{y}} - \bm{y}) > 0
  \end{equation}
  である.また,補題\ref{lemm:proj}より,
  \begin{equation}\label{eq:prj}
    (\bm{y} - \bar{\bm{y}})^{\mathrm{T}}(\bm{x} - \bar{\bm{y}}) \leq 0, \; x \in X
  \end{equation}
  が成立する.(\ref{eq:norm})式,(\ref{eq:prj})式より,
  \begin{equation}
    (\bar{\bm{y}} - \bm{y})^{\mathrm{T}} \bm{x} \geq (\bar{\bm{y}} - \bm{y})^{\mathrm{T}} \bar{\bm{y}} > (\bar{\bm{y}} - \bm{y})^{\mathrm{T}} \bm{y}, \; x \in X \nonumber
  \end{equation}
  を得る.ここで,$\bm{a} = \bar{\bm{y}} - \bm{y}, \, b = (\bar{\bm{y}} - \bm{y})^{\mathrm{T}} \bar{\bm{y}}$と置くと,$\bar{\bm{y}}$の一意性より補題\ref{lemm:hyperplane1}が示される.\qed
\end{proof}

\begin{proof}[補題\ref{lemm:hyperplane2}の証明]

\end{proof}

\begin{proof}[補題\ref{lemm:hyperplane3}の証明]

\end{proof}

\begin{proof}[定理\ref{theo:epi}の証明]

\end{proof}

\begin{proof}[定理\ref{theo:quasi}の証明]

\end{proof}

\begin{proof}[定理\ref{theo:convex}の証明]

\end{proof}

\begin{proof}[定理\ref{theo:convex2}の証明]

\end{proof}

\end{document}
