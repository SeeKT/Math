\documentclass{jsreport}
\usepackage{graphicx, url, algorithm, algorithmic, float, booktabs, listings, color, pdfpages, amsmath, amssymb, latexsym, mathtools, ascmac, amsfonts, amsthm}
\lstset{
  basicstyle={\ttfamily},
  identifierstyle={\small},
  commentstyle={\smallitshape},
  keywordstyle={\small\bfseries},
  ndkeywordstyle={\small},
  stringstyle={\small\ttfamily},
  frame={tb},
  breaklines=true,
  columns=[l]{fullflexible},
  numbers=left,
  xrightmargin=0zw,
  xleftmargin=3zw,
  numberstyle={\scriptsize},
  stepnumber=1,
  numbersep=1zw,
  lineskip=-0.5ex
}
\usepackage[table,xcdraw]{xcolor}
\newtheorem{theo}{定理}[chapter]
\newtheorem{defi}{定義}[chapter]
\newtheorem{lemm}{補題}[chapter]
\newtheorem{prop}{命題}[chapter]
\newtheorem{coro}{系}[chapter]
\newcommand{\red}[1]{\textcolor{red}{#1}}
\newcommand{\blue}[1]{\textcolor{blue}{#1}}
\newcommand{\green}[1]{\textcolor{green}{#1}}
\renewcommand{\baselinestretch}{1.1}
\usepackage{mathrsfs}
\usepackage{bm}
\def\qed{\hfill $\Box$}
\usepackage{tikz}
\usetikzlibrary{intersections, calc, arrows}
\renewcommand\proofname{\bf 証明}

\begin{document}

\chapter{線形計画問題}
本章では,線形計画問題の定式化,最適性,およびアルゴリズムについて述べる.
\section{線形計画問題の例}
線形計画問題の例として,しばしば生産計画問題が挙げられる.例えば,次のような問題である.
\begin{description}
  \item[生産計画問題]
  A社では,2種類の製品$P_1, \, P_2$を同じ原料から生産している.$P_1$ 1トン生産するのに要する原料は2.5トン,電力は5kwh,労力は3人時,$P_2$ 1トン生産するのに要する原料は5トン,電力は6kwh,労力は2人時である.1日の原料,電力,労力の使用可能量は,それぞれ350トン,450kwh,240人時である.$P_1, \, P_2$ 1トン当たりの粗利益は,それぞれ,4万円,5万円である.このとき,総粗利益を最大にするには,$P_1, \, P_2$を何トンずつ生産すればよいか.
\end{description}

$P_1, \, P_2$の1日の生産量(トン)をそれぞれ$x_1, \, x_2$とする.この問題は表\ref{tab:product_plan}のように表される.
\begin{table}[htb]
  \centering
    \caption{生産計画問題}
    \begin{tabular}{c|c|c|c}
       & $P_1$ & $P_2$ &  \\ \hline
      原料 & $2.5x_1$ & $5x_2$ & $350$ \\ \hline
      電力 & $5x_1$ & $6x_2$ & $450$ \\ \hline
      労力 & $3x_1$ & $2x_2$ & $240$ \\ \hline
      粗利益 & $4x_1$ & $5x_2$ & \\
    \end{tabular}
    \label{tab:product_plan}
\end{table}

以上より,この生産計画問題は次の線形計画問題となる.
\begin{align}\label{eq:product_plan}
  \mathrm{maximize} \; \; &4x_1 + 5x_2 \nonumber \\
  \mathrm{subject \; to} \; \; &2.5x_1 + 5x_2 \leq 350 \nonumber \\
  &5x_1 + 6x_2 \leq 450 \\
  &3x_1 + 2x_2 \leq 240 \nonumber \\
  &x_1, x_2 \geq 0 \nonumber
\end{align}

\section{線形計画問題の最適解}
\ref{sec:convex_prob}節で述べたように,線形計画問題の実行可能領域は凸多面体であり,目的関数も線形(凸関数)であるので,線形計画問題は凸計画問題である.よって,任意の局所最適解は大域最適解である.

線形計画問題の最適解は,実行可能領域の多面体の頂点に存在する.この事実を確認する.
一般に多面体$X$は,すべての頂点からなる集合$V = \{\bm{v}_1, \bm{v}_2, \ldots, \bm{v}_p\}$を用いて,次のように表すことができる.
\begin{equation}
  X = \left\{\bm{x} \in \mathbb{R}^n \, | \, \bm{x} = \sum_{i = 1}^p k_i \bm{v}_i , \, \sum_{i = 1}^p k_i = 1, \, k_i \geq 0, \, i = 1, 2, \ldots, p \right\} \nonumber
\end{equation}

また,線形計画問題の目的関数は,$\bm{c}^{\mathrm{T}}\bm{x}$と表される.実行可能解$\bm{x} \in X$を考えると,$\sum_{i = 1}^p k_i = 1, \, k_i \geq 0, \, i = 1, 2, \ldots, p$を満たす$(k_1, k_2, \ldots, k_p)^{\mathrm{T}}$が存在して,
\begin{equation}
  \bm{x} = \sum_{i = 1}^p k_i \bm{v}_i \nonumber
\end{equation}
と表される.両辺に左から$\bm{c}^{\mathrm{T}}$を掛けると,
\begin{equation}
  \bm{c}^{\mathrm{T}} \bm{x} = \bm{c}^{\mathrm{T}} \left(\sum_{i = 1}^p k_i \bm{v}_i \right) = \sum_{i = 1}^p k_i \bm{c}^{\mathrm{T}} \bm{v}_i \nonumber
\end{equation}
となる.$\sum_{i = 1}^p k_i = 1, \, k_i \geq 0, \, i = 1, 2, \ldots, p$より,
\begin{equation}
  \min_{i = 1, 2, \ldots, p} \bm{c}^{\mathrm{T}} \bm{v}_i \leq \bm{c}^{\mathrm{T}} \bm{x} \leq \max_{i = 1, 2, \ldots, p} \bm{c}^{\mathrm{T}} \bm{v}_i \nonumber
\end{equation}
となる.$i = 1, 2, \ldots, p$について,$\bm{v}_i \in X$より,実行可能領域$X$での$\bm{c}^{\mathrm{T}}\bm{x}$の最大値および最小値を与える解の1つは,頂点集合$V$内にある.

\section{標準形の線形計画問題}
\subsection{線形計画問題の標準形}
変数が非負で等式制約条件しかもたない最小化問題を標準形(standard form)の線形計画問題という.すなわち,線形計画問題の標準形は,
\begin{align}\label{eq:standard}
  \mathrm{minimize} \; \; &\bm{c}^{\mathrm{T}}\bm{x} \nonumber \\
  \mathrm{subject \; to} \; \; &A\bm{x} = \bm{b} \\
  &\bm{x} \geq \bm{0} \nonumber
\end{align}
である.ただし,$\bm{c} = (c_1, c_2, \ldots, c_n)^{\mathrm{T}}, \, \bm{b} = (b_1, b_2, \ldots, b_m)^{\mathrm{T}}$はそれぞれ$n$次元,$m$次元の実数定ベクトルである.$A$は$m \times n \, (m \leq n)$の実数定行列である.$\bm{x} = (x_1, x_2, \ldots, x_n)^{\mathrm{T}}$は$n$次元の実数変数ベクトルであり,決定変数ベクトルと呼ばれる.

\subsection{標準形への変形}
\begin{description}
  \item[最大化問題の場合]
  線形計画問題が最大化問題の場合,つまり,
  \begin{equation}
    \mathrm{maximize} \; \; \bm{c}^{\mathrm{T}}\bm{x} \nonumber
  \end{equation}
  の場合,$\bm{d} = -\bm{c}$とすると,
  \begin{equation}
    \mathrm{maximize} \; \; \bm{c}^{\mathrm{T}}\bm{x} \Leftrightarrow \mathrm{minimize} \; \; \bm{d}^{\mathrm{T}}\bm{x} \nonumber
  \end{equation}
  と最小化問題に変形できる.
  \item[不等式条件$\leq$の場合]
  制約条件の1つが,
  \begin{equation}
    \bm{a}_i^{\mathrm{T}}\bm{x} \leq b_i \nonumber
  \end{equation}
  である場合,スラック変数(slack variable) $s \, (s \geq 0)$を導入して,
  \begin{equation}
    \bm{a}_i^{\mathrm{T}}\bm{x} \leq b_i \Leftrightarrow \bm{a}_i^{\mathrm{T}}\bm{x} + s = b_i, \; s \geq 0 \nonumber
  \end{equation}
  と等式条件に変形される.
  \item[不等号条件$\geq$の場合]
  制約条件の1つが,
  \begin{equation}
    \bm{a}_j^{\mathrm{T}}\bm{x} \geq b_j \nonumber
  \end{equation}
  である場合,サープラス変数(surplus variable) $t \, (t \geq 0)$を導入して,
  \begin{equation}
    \bm{a}_j^{\mathrm{T}}\bm{x} \leq b_j \Leftrightarrow \bm{a}_j^{\mathrm{T}}\bm{x} - t = b_j, \; t \geq 0 \nonumber
  \end{equation}
  と等式条件に変形される.
  \item[自由変数(非負条件の無い変数)が存在する場合]
  $x_k$に対して,$x_k \geq 0$という条件が存在せず,負の値も取る場合は,$x_k$を次の条件を満たす2つの変数$x_k^{+}$と$x_k^{-}$に分解して取り扱われる.
  \begin{equation}
    x_k = x_k^{+} - x_k^{-}, \; x_k^{+} \geq 0, x_k^{-} \geq 0 \nonumber
  \end{equation}
\end{description}

例えば,最適化問題(\ref{eq:product_plan})を標準形の線形計画問題に変形すると,(\ref{eq:product_plan_stand})になる.
\begin{align}\label{eq:product_plan_stand}
  \mathrm{minimize} \; \; &-4x_1 - 5x_2 \nonumber \\
  \mathrm{subject \; to} \; \; &2.5x_1 + 5x_2 + x_3 = 350 \nonumber \\
  &5x_1 + 6x_2 + x_4 = 450 \\
  &3x_1 + 2x_2 + x_5 = 240 \nonumber \\
  &x_1, x_2, x_3, x_4, x_5 \geq 0 \nonumber
\end{align}

\section{基底解と最適解}
\subsection{用語}
\begin{description}
  \item[基底行列]
  基底行列(basic matrix) $B$とは,制約条件の係数行列$A$の$m$個(制約条件数)の列ベクトルから作られる$m \times m$の正則行列のことをいう($|B| \neq 0$).
  \item[基底解]
  線形計画問題の基底解(basic solution)とは,ある基底行列$B$で選ばれなかった$(n - m)$個の列に対応する変数(非基底変数(nonbasic variable))を0とおき,選ばれた$m$個の変数(基底変数(basic variable))に対する正則な連立方程式
  \begin{equation}\label{eq:basic}
    B \left(
    \begin{array}{c}
      x_1^{B} \\
      x_2^{B} \\
      \vdots \\
      x_m^{B}
    \end{array}
    \right) = \bm{b}
  \end{equation}
  を解くことにより得られる一意的なベクトル$B^{-1}\bm{b}$のことをいう.
  \item[実行可能基底解]
  実行可能解であり,基底解である解を実行可能基底解(basic feasible solution)という.基底行列を$B$とすると,$B^{-1}\bm{b} \geq \bm{0}$のときに基底行列$B$に対応する基底解は実行可能基底解となる.
  \item[退化]
  基底解では,$m$個の基底変数のみが正の値を取りうるが,この$m$個の基底変数の内少なくとも1つが0となる基底解は,退化しているという.
\end{description}

\subsection{線形計画問題の基本定理}
標準形の線形計画問題(\ref{eq:standard})において,次の定理が成立する.
\begin{theo}[線形計画問題の基本定理]\label{theo:linear_th}
  標準形の線形計画問題(\ref{eq:standard})が与えられたとき,次の1, 2が成立する.
  \begin{enumerate}
    \item 実行可能解が存在するならば,必ず実行可能基底解が存在する.
    \item 最適解が存在するならば,実行可能基底解の中に最適解が存在する.
  \end{enumerate}
\end{theo}
定理\ref{theo:linear_th}より,線形計画問題の最適解を求めるには,基底解の中で,実行可能かつ最適なものを見つければよい\footnote{実行可能領域の端点を辿っていけば最適解に到達する.後述のシンプレックス法はこの考えに基づいている.}.

\subsection{基底形式}
標準形の線形計画問題
\begin{align}\label{eq:standard_re}
  \mathrm{minimize} \; \; &z = \bm{c}^{\mathrm{T}}\bm{x} \nonumber \\
  \mathrm{subject \; to} \; \; &A\bm{x} = \bm{b} \\
  &\bm{x} \geq \bm{0} \nonumber
\end{align}
において,$m \times n$行列$A$から$m$個の列を選び基底行列$B$とする.ただし,$B$は正則行列とする.残りの列からなる行列を$N$とする.列の順番と変数の順番を適当に変えることにより,
$A = (B, N), \bm{x} = (\bm{x}_B^{\mathrm{T}}, \bm{x}_{N}^{\mathrm{T}})^{\mathrm{T}}, \bm{c} = (\bm{c}_B^{\mathrm{T}}, \bm{c}_N^{\mathrm{T}})^{\mathrm{T}}$
と表すと,式(\ref{eq:standard_re})は,
\begin{align}\label{eq:standard_re_basic}
  \mathrm{minimize} \; \; &z = \bm{c}_B^{\mathrm{T}}\bm{x}_B + \bm{c}_N^{\mathrm{T}}\bm{x}_N \nonumber \\
  \mathrm{subject \; to} \; \; &B\bm{x}_B + N\bm{x}_N = \bm{b} \\
  &\bm{x}_B, \bm{x}_N \geq \bm{0} \nonumber
\end{align}
と表される.$\bm{x}_B$は基底変数ベクトル,$\bm{x}_N$は非基底変数ベクトル,
$\bm{x}_B$の各成分$x_j^B$は基底変数,
$\bm{x}_N$の各成分$x_j^N$は非基底変数と呼ばれる.$B$は正則より,目的関数から$\bm{x}_B$を消去すると,(\ref{eq:standard_re_basic})の問題は,
\begin{align}\label{eq:standard_re_equal}
  \mathrm{minimize} \; \; &z = \bm{c}_B^{\mathrm{T}}B^{-1}\bm{b} + (\bm{c}_N^{\mathrm{T}} - \bm{c}_B^{\mathrm{T}}B^{-1}N)\bm{x}_N \nonumber \\
  \mathrm{subject \; to} \; \; &\bm{x}_B + B^{-1}N\bm{x}_N = B^{-1}\bm{b} \\
  &\bm{x}_B, \bm{x}_N \geq \bm{0} \nonumber
\end{align}
と等価になる.(\ref{eq:standard_re_equal})の一部
\begin{align}\label{eq:basic_form}
  &z = \bm{c}_B^{\mathrm{T}}B^{-1}\bm{b} + (\bm{c}_N^{\mathrm{T}} - \bm{c}_B^{\mathrm{T}}B^{-1}N)\bm{x}_N \nonumber \\
  &\bm{x}_B + B^{-1}N\bm{x}_N = B^{-1}\bm{b}
\end{align}
は,基底形式(basic form)あるいは正準形(cannoncial form)と呼ばれる.このとき,基底行列は$B$であり,対応する基底解は
\begin{equation}\label{eq:basic_sol}
  \bm{x} = \left(
  \begin{array}{c}
    \bm{x}_B \\
    \bm{x}_N
  \end{array}
  \right) = \left(
  \begin{array}{c}
    B^{-1}\bm{b} \\
    \bm{0}
  \end{array}
  \right)
\end{equation}
となる.
以後,$\bar{\bm{c}}^{\mathrm{T}} = \bm{c}_N^{\mathrm{T}} - \bm{c}_B^{\mathrm{T}}B^{-1}N, \, \bar{N} = B^{-1}N, \, \bar{\bm{b}} = B^{-1}\bm{b}$と定める.
また,$\bm{c}_N^{\mathrm{T}}, \, \bar{N}, \, \bar{\bm{b}}$の各成分をそれぞれ,
$\bar{c}_{j}, \, \bar{a}_{ij}, \, \bar{b}_i$と表す.

(\ref{eq:basic_form})式を$\bar{\bm{c}}, \bar{N}, \bar{\bm{b}}$を用いて表すと,
\begin{align}\label{eq:basic_form_re}
  &z = \bm{c}_B^{\mathrm{T}}\bar{\bm{b}} + \bar{\bm{c}}^{\mathrm{T}}\bm{x}_N \nonumber \\
  &\bm{x}_B + \bar{N}\bm{x}_N = \bar{\bm{b}}
\end{align}
となる.(\ref{eq:standard_re_equal})式において,
\begin{equation}\label{eq:feasible_b}
  \bar{\bm{b}} = B^{-1}\bm{b} \geq \bm{0} \; \; (実行可能性)
\end{equation}
が成立すれば,(\ref{eq:basic_sol})式の基底解は実行可能解となる.(\ref{eq:feasible_b})が満たされると仮定する.

基底形式(\ref{eq:basic_form})のある非基底変数$x_r^N$の係数$\bar{c}_r$が負であれば,この非基底変数に正の値を取らせることにより,目的関数が改善できる\footnote{$\bar{c}_r < 0$ならば,$x_r^N > 0$を取ることで,実行可能性を維持したまま(\ref{eq:basic_form})式の$z$の値を小さくすることができる.}.逆に,すべての非基底変数$x_j^N$の係数$\bar{c}_j$が正,すなわち,
\begin{equation}\label{eq:opt_con}
  \bar{\bm{c}}^{\mathrm{T}} = \bm{c}_N^{\mathrm{T}} - \bm{c}_B^{\mathrm{T}}B^{-1}N \geq \bm{0}^{\mathrm{T}} \; \; (最適性)
\end{equation}
が成立すれば,この実行可能基底解は最適解の1つである.

次に,$\bar{c}_r$が負であるとき,実行可能性を維持する,つまり,$x_r^N$に正の値を取らせることが可能であるかどうかを考える.ここで,$x_r^N$以外の非基底変数$x_j^N$の値を0に固定する.このとき,
\begin{equation}
  \bm{x}_B = \bar{\bm{b}} -
  \left(
  \begin{array}{c}
    \bar{a}_{1r} \\
    \bar{a}_{2r} \\
    \vdots \\
    \bar{a}_{mr}
  \end{array}
  \right)x_r^N
\end{equation}
を満たせば,制約条件の最初の等式条件$\bm{x}_B + \bar{N}\bm{x}_N = \bar{\bm{b}}$を満足する.
実際,$\bm{x}_N = \left(0, \cdots, x_r^N, \cdots, 0 \right)^{\mathrm{T}}$
のとき,
\begin{equation}
  \bar{N} \bm{x}_N = \left(
  \begin{array}{ccccc}
    \bar{a}_{11} & \cdots & \bar{a}_{1r} & \cdots & \bar{a}_{1 n-m} \\
    \vdots & \ddots & \vdots & \ddots & \vdots \\
    \bar{a}_{m1} & \cdots & \bar{a}_{mr} & \cdots & \bar{a}_{m n-m}
  \end{array}
  \right) \left(
  \begin{array}{c}
    0 \\
    \vdots \\
    x_r^N \\
    \vdots \\
    0
  \end{array}
  \right) = \left(
  \begin{array}{c}
    \bar{a}_{1r} \\
    \bar{a}_{2r} \\
    \vdots \\
    \bar{a}_{mr}
  \end{array}
  \right)x_r^N \nonumber
\end{equation}
を満たす.さらに,非負条件を満たすには,$\bm{x}_B \geq \bm{0}$より,
\begin{equation}\label{eq:notminus}
  \bar{\bm{b}} -
  \left(
  \begin{array}{c}
    \bar{a}_{1r} \\
    \bar{a}_{2r} \\
    \vdots \\
    \bar{a}_{mr}
  \end{array}
  \right)x_r^N \geq \bm{0}
\end{equation}
が成立すればよい.

\end{document}
