\documentclass{jsreport}
\usepackage{graphicx, url, algorithm, algorithmic, float, booktabs, listings, color, pdfpages, amsmath, amssymb, latexsym, mathtools, ascmac, amsfonts, amsthm}
\lstset{
  basicstyle={\ttfamily},
  identifierstyle={\small},
  commentstyle={\smallitshape},
  keywordstyle={\small\bfseries},
  ndkeywordstyle={\small},
  stringstyle={\small\ttfamily},
  frame={tb},
  breaklines=true,
  columns=[l]{fullflexible},
  numbers=left,
  xrightmargin=0zw,
  xleftmargin=3zw,
  numberstyle={\scriptsize},
  stepnumber=1,
  numbersep=1zw,
  lineskip=-0.5ex
}
\usepackage[table,xcdraw]{xcolor}
\newtheorem{theo}{定理}[chapter]
\newtheorem{defi}{定義}[chapter]
\newtheorem{lemm}{補題}[chapter]
\newtheorem{prop}{命題}[chapter]
\newtheorem{coro}{系}[chapter]
\newcommand{\red}[1]{\textcolor{red}{#1}}
\newcommand{\blue}[1]{\textcolor{blue}{#1}}
\newcommand{\green}[1]{\textcolor{green}{#1}}
\renewcommand{\baselinestretch}{1.1}
\usepackage{mathrsfs}
\usepackage{bm}
\def\qed{\hfill $\Box$}
\usepackage{tikz}
\usetikzlibrary{intersections, calc, arrows}
\renewcommand\proofname{\bf 証明}

\begin{document}
\chapter{数学的準備}
\section{諸定義}
$n$次元実数空間$\mathbb{R}^n$を考える.
\begin{itemize}
  \item $\varepsilon$-近傍
  \begin{itemize}
    \item $B(x, \varepsilon) = \{\bm{y} \in \mathbb{R}^n \, | \, \|\bm{y} - \bm{x} \| < \varepsilon\}$
    \item ただし,$\|\cdot\|$はユークリッドノルムであり,
    $\|\bm{x}\| = (\bm{x}^{\mathrm{T}}\bm{x})^{1/2}$である.
  \end{itemize}
  \item $X \subseteq \mathbb{R}^n$が開集合(open set)である
  \begin{itemize}
    \item $\forall \bm{x} \in X, \, \exists \varepsilon > 0; \, B(\bm{x}, \varepsilon) \subseteq X$
  \end{itemize}
  \item $X \subseteq \mathbb{R}^n$が閉集合(closed set)である
  \begin{itemize}
    \item $X$の補集合$X^\mathrm{c}$が開集合である.
  \end{itemize}
  \item $\bm{x}$が$X \subseteq \mathbb{R}^n$の内点(interior point)である
  \begin{itemize}
    \item $\bm{x} \in X \subseteq \mathbb{R}^n$に対して,
    $\exists \varepsilon > 0; \, B(\bm{x}, \varepsilon) \subseteq X$が成立する.
  \end{itemize}
  \item $X$の内部(interior),$\mathrm{Int}(X)$
  \begin{itemize}
    \item $X \subseteq \mathbb{R}^n$の内点の集合.
  \end{itemize}
  \item $\bm{x}$が$X \subseteq \mathbb{R}^n$の触点(contact point)である
  \begin{itemize}
    \item $\bm{x} \in X \subseteq \mathbb{R}^n$に対して,$\forall \varepsilon > 0; \, B(\bm{x}, \varepsilon) \cap X \neq \emptyset$が成立する.
  \end{itemize}
  \item $X$の閉包(closure),$\mathrm{Cl}(X)$
  \begin{itemize}
    \item $X \subseteq \mathbb{R}^n$の触点の集合.
    \item $X$を含む最小の閉集合.
  \end{itemize}
  \item $X$の境界(boundary),$\mathrm{Bd}(X)$
  \begin{itemize}
    \item $\mathrm{Bd}(X) = \mathrm{Cl}(X) \setminus \mathrm{Int}(X)$.
  \end{itemize}
  \item $\bm{x}$は$X$の集積点である
  \begin{itemize}
    \item 任意の$\varepsilon > 0$に対して,$B(\bm{x}, \varepsilon) \cap X$が$\bm{x}$と異なる要素を含む.
  \end{itemize}
  \item 孤立点(isolated point)
  \begin{itemize}
    \item $X$の集積点でない$X$の触点
  \end{itemize}
  \item $X$が有界である(bounded)
  \begin{itemize}
    \item $\exists \varepsilon > 0; \, X \subseteq B(\bm{0}, \varepsilon)$.
  \end{itemize}
  \item 収束
  \begin{itemize}
    \item 点列$\{\bm{x}^i\}, \, i = 1, 2, \ldots$を考える.
    $\forall \varepsilon > 0, \, \exists I_{\varepsilon}; \; \|\bm{x}^i - \bm{x}\| < \varepsilon, \, i \geq I_{\varepsilon}$となる点$\bm{x}$が存在するとき,$\bm{x}$は点列$\{\bm{x}^i\}$の極限(limit)といい,点列$\{\bm{x}^i\}$は$\bm{x}$に収束する(converge)という.
  \end{itemize}
  \item 点列$\{\bm{x}^i\}$の集積点
  \begin{itemize}
    \item $\{\bm{x}^i\}$の部分点列$\{\bm{x}^{l_i}\}$が点$\bm{x}$に収束するとき,点$\bm{x}$を点列$\{\bm{x}^i\}$の集積点という.
  \end{itemize}
\end{itemize}

有界閉集合$X \subseteq \mathbb{R}^n$について,$\{\bm{x}^i\} \subseteq X$なる無限点列は少なくとも1つの集積点をもつ.

\section{関数}
\begin{itemize}
  \item 連続性
  \begin{itemize}
    \item 関数$\bm{f}: X \rightarrow \mathbb{R}^m$を考える($X \subseteq R^n$).
    \begin{equation}
      \forall \epsilon > 0, \exists \delta > 0; \, \|\bm{x} - \bm{x}^0\| < \delta \Rightarrow \|\bm{f}(\bm{x}) - \bm{f}(\bm{x}^0) \| < \varepsilon \nonumber
    \end{equation}
    が成立するとき,$\bm{f}$は点$\bm{x}^0$で連続(continuous)であるという.
    \item 任意の$\bm{x} \in X$で連続となるとき,関数$\bm{f}$は$X$で連続という.
  \end{itemize}
  \item 実数値関数
  \begin{itemize}
    \item 値域が実数集合の関数$f: X \rightarrow \mathbb{R}$.
  \end{itemize}
  \item 実数値関数のクラス
  \begin{itemize}
    \item $f: X \rightarrow \mathbb{R}$($X \subseteq \mathbb{R}^n$は開集合)を考える.
    \item $f$が連続であるとき,$X$上で$C^0$級と呼ばれ,$f \in C^0$と記す.
    \item $f \in C^0$で,$\partial f(\bm{x}) / \partial x_i, \, i = 1, 2, \ldots, n$が存在し,連続であれば,$f$は$X$上で$C^1$級と呼ばれ,$f \in C^1$と表す.
    \item $f \in C^1$で,$\partial^2 f(\bm{x}) / \partial x_i \partial x_j, \, i,j = 1, 2, \ldots, n$が存在し,連続であれば,$f$は$X$上で$C^2$級と呼ばれ,$f \in C^2$と表す.
    \item 以後$C^k$級も同様に定義される.
  \end{itemize}
  \item 勾配ベクトル$\nabla f(\bm{x})$
    \begin{equation}
      \nabla f(\bm{x}) = \left(\frac{\partial f(\bm{x})}{\partial x_1}, \frac{\partial f(\bm{x})}{\partial x_2}, \cdots, \frac{\partial f(\bm{x})}{\partial x_n}\right) \nonumber
    \end{equation}
  \item ヘッセ行列$H(\bm{x}) = \nabla^2 f(\bm{x})$
  \begin{equation}
    \nabla^2 f(\bm{x}) = \left(
    \begin{array}{cccc}
      \displaystyle \frac{\partial^2 f(\bm{x})}{{\partial x_1}^2} & \displaystyle \frac{\partial^2 f(\bm{x})}{\partial x_1 \partial x_2} & \displaystyle \cdots & \displaystyle \frac{\partial^2 f(\bm{x})}{\partial x_1 \partial x_n} \\
      \displaystyle \frac{\partial^2 f(\bm{x})}{\partial x_2 \partial x_1} & \displaystyle \frac{\partial^2 f(\bm{x})}{{\partial x_2}^2} & \displaystyle \cdots & \displaystyle \frac{\partial^2 f(\bm{x})}{\partial x_2 \partial x_n} \\
      \vdots & \vdots & \ddots & \vdots \\
      \displaystyle \frac{\partial^2 f(\bm{x})}{\partial x_n \partial x_1} & \displaystyle \frac{\partial^2 f(\bm{x})}{\partial x_n \partial x_2} & \displaystyle \cdots & \displaystyle \frac{\partial^2 f(\bm{x})}{{\partial x_n}^2}
    \end{array}
    \right) \nonumber
  \end{equation}
\end{itemize}

\paragraph{Weierstrassの定理}
有界閉集合$X \subseteq \mathbb{R}^n$上の連続な実数値関数$f(\bm{x})$は$X$内の点で最大値,最小値をとる.

\paragraph{平均値の定理}
$f: X \rightarrow \mathbb{R} \, (X \subseteq \mathbb{R}^n), \, f \in C^1, \, \bm{x}^1, \bm{x}^2 \in X$に対して,
\begin{equation}
  f(\bm{x}^1) = f(\bm{x}^2) + \nabla f(\theta \bm{x}^1 + (1 - \theta) \bm{x}^2)^{\mathrm{T}}(\bm{x}^1 - \bm{x}^2) \nonumber
\end{equation}
を満たす$\theta \in (0, 1)$が存在する.

(注)関数$f: I \rightarrow \mathbb{R} \, (I \subseteq \mathbb{R}), \, f \in C^1, \, x_1, x_2 \in I$に対しては,
\begin{equation}
  f(x_1) = f(x_2) + f^{\prime}(\theta x_1 + (1-\theta)x_2)(x_1 - x_2) \nonumber
\end{equation}
を満たす$\theta \in (0, 1)$が存在する.これは,点$x_1, x_2 \in I$を結ぶ線分内に,$f(x_1), f(x_2)$を結ぶ直線と同じ傾きとなる接線を持つ点$c = \theta x_1 + (1-\theta)x_2, \, \theta \in (0, 1)$が存在することに相当する.

\paragraph{Taylorの定理}
$f: X \rightarrow \mathbb{R} \, (X \subseteq \mathbb{R}^n), \, f \in C^2, \, \bm{x}^1, \bm{x}^2 \in X$に対して,
\begin{equation}
  f(\bm{x}^1) = f(\bm{x}^2) + \nabla f(\bm{x}^2)^{\mathrm{T}} (\bm{x}^1 - \bm{x}^2)^{\mathrm{T}} + \frac{1}{2} (\bm{x}^1 - \bm{x}^2)^{\mathrm{T}} \nabla^2 f(\bm{x}^2) (\bm{x}^1 - \bm{x}^2) + o(\|\bm{x}^1 - \bm{x}^2 \|^2) \nonumber
\end{equation}
が成立する.ただし,$o$は$\lim\limits_{t \rightarrow 0} o(t) / t = 0$なる関数である.
同様に,$f \in C^1$に対して,
\begin{equation}
    f(\bm{x}^1) = f(\bm{x}^2) + \nabla f(\bm{x}^2)^{\mathrm{T}} (\bm{x}^1 - \bm{x}^2)^{\mathrm{T}} + o(\|\bm{x}^1 - \bm{x}^2 \|^1) \nonumber
\end{equation}
が成立する.

(注)関数$f: I \rightarrow \mathbb{R} \, (I \subseteq \mathbb{R}), \, f \in C^1, \, x_1, x_2 \in I$に対しては,
\begin{equation}
  f(x_1) = f(x_2) + f^{\prime}(x_2)(x_1 - x_2) + \frac{f^{\prime \prime}(x_2)}{2}(x_1 - x_2)^2 + o((x_1 - x_2)^2) \nonumber
\end{equation}
\begin{equation}
  f(x_1) = f(x_2) + f^{\prime}(x_2)(x_1 - x_2) + o(x_1 - x_2) \nonumber
\end{equation}
が成立する.$o$を用いずに表すと,
\begin{equation}
  f(x_1) = f(x_2) + f^{\prime}(x_2)(x_1 - x_2) + \frac{f^{\prime \prime}(x_2)}{2}(x_1 - x_2)^2 + \cdots + \frac{f^{(n - 1)}(x_2)}{(n - 1)!}(x_1 - x_2)^{n - 1} + R_n \nonumber
\end{equation}
\begin{equation}
  R_n \coloneqq \frac{f^{(n)}(\theta x_1 + (1 - \theta)x_2)}{n!}(x_1 - x_2)^n \nonumber
\end{equation}
を満たす$\theta \in (0, 1)$が存在する.$n = 1$のときは平均値の定理.つまり,Taylorの定理は平均値の定理の一般化と考えることができる.

\paragraph{陰関数の定理}
$h_i: \mathbb{R}^n \rightarrow \mathbb{R}, \, \bm{x}^0 = (x_1^0, x_1^0, \ldots, x_n^0)^{\mathrm{T}} \in \mathbb{R}^n$の近傍で,$h_i \in C^p \, (p \geq 1)$
\begin{equation}
  h_i(\bm{x}^0) = 0, \, i = 1, 2, \ldots, m \nonumber
\end{equation}
が成立するとする.このとき,ヤコビ行列(Jacobian matrix)
\begin{equation}
  J(\bm{x}^0) = \left(
  \begin{array}{ccc}
    \displaystyle \frac{\partial h_1(\bm{x}^0)}{\partial x_1} & \cdots & \displaystyle \frac{\partial h_1(\bm{x}^0)}{\partial x_m} \\
    \vdots & \vdots & \vdots \\
    \displaystyle \frac{\partial h_m(\bm{x}^0)}{\partial x_1} & \cdots & \displaystyle \frac{\partial h_m(\bm{x}^0)}{\partial x_m}
  \end{array}
  \right) \nonumber
\end{equation}
が正則ならば,ある$\varepsilon > 0$に対して$\hat{\bm{x}}^0 = (x_{m + 1}^0, \ldots, x_n^0)^{\mathrm{T}} \in \mathbb{R}^{n - m}$の近傍$U = B(\hat{\bm{x}}^0, \varepsilon)$が存在し,$\hat{\bm{x}} \in U$に対して,
\begin{enumerate}
  \item $\phi_i \in C^p, \, i = 1, 2, \ldots, m$
  \item $x_i = \phi(\hat{\bm{x}}), \, i = 1, 2, \ldots, m$
  \item $h_i(\phi_1(\hat{\bm{x}}), \phi_2(\hat{\bm{x}}), \ldots, \phi_m(\hat{\bm{x}}), \hat{\bm{x}}) = 0, i = 1, 2, \ldots, m$
\end{enumerate}
となる陰関数(implicit function)$\phi_i, \, i = 1, 2, \ldots, m$が存在する\footnote{解析学の教科書では,陰関数が唯一存在すると記されている.}.

(注) 2変数関数を考える.関数$f: \Omega \rightarrow \mathbb{R} \, (\Omega \subseteq \mathbb{R}^2), \, f \in C^p, \, p \geq 1, \, (x_1, x_2) \in \Omega$とする.$f(x_1, x_2) = 0$のとき,$x = x_1$を含む開区間$I$および$I$上で定義された$C^p$級の関数$\phi$がただ1つ存在し,$\phi(x_1) = x_2$および
\begin{equation}
  u(x) \coloneqq f(x, \phi(x)) = 0 \; \; (\forall x \in I) \nonumber
\end{equation}
が成立する.

\section{凸集合}
$X \subseteq \mathbb{R}^n$とする.$\forall \bm{x}^1, \bm{x}^2 \in X, \, \forall \lambda \in [0, 1]$に対して,$\lambda \bm{x}^1 + (1 - \lambda)\bm{x}^2 \in X$が成立するとき,$X$は凸集合(convex set)である
\footnote{2点$\bm{x}^1, \bm{x}^2$を結ぶ線分上の任意の内分点は,$x_{\lambda} = \lambda \bm{x}^1 + (1 - \lambda)\bm{x}^2, \, \lambda \in [0, 1]$と書ける.}.

\begin{lemm}\label{lemm:convex}
  $X_1, X_2, \ldots \subseteq \mathbb{R}^n$を凸集合とすると,$\bigcap \limits_{i = 1, 2, \ldots} X_i$も凸集合となる.
\end{lemm}

\begin{lemm}\label{lemm:hyperplane1}
  $X \subseteq \mathbb{R}^n$を空でない閉凸集合,$\bm{y} \notin X$とする.このとき,
  \begin{equation}
    \bm{a}^{\mathrm{T}}\bm{x} \geq b > \bm{a}^{\mathrm{T}}\bm{y}, \; \forall x \in X \nonumber
  \end{equation}
  となる分離超平面(separating hyperplane)$\{\bm{x} \in \mathbb{R}^n \, | \, \bm{a}^{\mathrm{T}} \bm{x} = b\} \; (\bm{a} \neq \bm{0})$が存在する.
\end{lemm}


\begin{lemm}\label{lemm:hyperplane2}
  $X \subseteq \mathbb{R}^n$を空でない凸集合,$\bm{y} \in \mathrm{Bd}(X)$とする.このとき,
  \begin{equation}
    \bm{a}^{\mathrm{T}} \bm{x} \geq \bm{a}^{\mathrm{T}} \bm{y}, \, \forall \bm{x} \in X \nonumber
  \end{equation}
  なる$\bm{a} \neq \bm{0}$が存在する.
\end{lemm}

\begin{lemm}\label{lemm:hyperplane3}
  $X, Y \subseteq \mathbb{R}^n$を空でない凸集合,$X \cap Y = \emptyset$とする.このとき,
  \begin{align}\label{eq:hyperplane3}
    \bm{a}^{\mathrm{T}}\bm{x} &\geq b, \; \forall \bm{x} \in X, \nonumber \\
    \bm{a}^{\mathrm{T}}\bm{y} &\leq b, \; \forall \bm{y} \in Y
  \end{align}
  なる$\bm{a} \neq \bm{0}, \, b$が存在する.
\end{lemm}

\section{凸関数}
$X \subseteq \mathbb{R}^n$を空でない凸集合とする.関数$f: X \rightarrow \mathbb{R}$を考える.任意の$\bm{x}^1, \bm{x}^2 \in X$,任意の$\lambda \in [0, 1]$に対して,
\begin{equation}
  f(\lambda \bm{x}^1 + (1 - \lambda)\bm{x}^2) \leq \lambda f(\bm{x}^1) + (1 - \lambda)f(\bm{x}^2) \nonumber
\end{equation}
が成立するとき,$f$は凸関数(convex function)であるという.また,任意の$\bm{x}^1, \bm{x}^2 \in X \, (\bm{x}^1 \neq \bm{x}^2)$,任意の$\lambda \in (0, 1)$に対して,
\begin{equation}
  f(\lambda \bm{x}^1 + (1 - \lambda)\bm{x}^2) < \lambda f(\bm{x}^1) + (1 - \lambda)f(\bm{x}^2) \nonumber
\end{equation}
が成立するとき,$f$は強意の凸関数(strictly convex function)であるという.さらに,任意の$\bm{x}^1, \bm{x}^2 \in X$,任意の$\lambda \in [0, 1]$に対して,
\begin{equation}
  f(\lambda\bm{x}^1 + (1 - \lambda)\bm{x}^2) \leq \max(f(\bm{x}^1), f(\bm{x}^2)) \nonumber
\end{equation}
が成立するとき,$f$は準凸関数(quasi-convex function)であるという.また,任意の$\bm{x}^1, \bm{x}^2 \in X \, (\bm{x}^1 \neq \bm{x}^2)$,任意の$\lambda \in (0, 1)$に対して,
\begin{equation}
  f(\lambda\bm{x}^1 + (1 - \lambda)\bm{x}^2) < \max(f(\bm{x}^1), f(\bm{x}^2)) \nonumber
\end{equation}
が成立するとき,$f$は強意の準凸関数(strictly quasi-convex function)であるという.
$-f$が凸関数,強意の凸関数,準凸関数,強意の準凸関数であるとき,それぞれ,$f$は凹関数(concave function),強意の凹関数,準凹関数,強意の準凹関数であるという.

\begin{theo}\label{theo:epi}
  $X \subseteq \mathbb{R}^n$を空でない凸集合とする.関数$f: X \rightarrow \mathbb{R}$が凸関数であるための必要十分条件は,$\mathbb{R}^{n + 1}$の部分集合,
  \begin{equation}
    \mathrm{epi}f = \{(\bm{x}, r) \, | \, \bm{x} \in X, r \geq f(\bm{x}), r \in \mathbb{R}\} \nonumber
  \end{equation}
  が凸集合となることである.$\mathrm{epi}f$を関数$f$のエピグラフ(epigraph)という.
\end{theo}

\begin{theo}\label{theo:quasi}
  $X \subseteq \mathbb{R}^n$を空でない凸集合とする.関数$f: X \rightarrow \mathbb{R}$が準凸関数であるための必要十分条件は,$\mathbb{R}^{n}$の部分集合,
  \begin{equation}
    L(r) = \{\bm{x} \in \mathbb{R}^n \, | \, f(\bm{x}) \leq r \} \nonumber
  \end{equation}
  が任意の$r \in \mathbb{R}$に対して凸集合となることである.$L(r)$を関数$f$のレベル集合(level set)という.
\end{theo}

\begin{theo}\label{theo:convex}
  $X \subseteq \mathbb{R}^n$を空でない開凸集合とする.関数$f: X \rightarrow \mathbb{R}$が微分可能とする.$f$が凸関数であるための必要十分条件は,
  \begin{equation}
    f(\bm{x}^2) \geq f(\bm{x}^1) + \nabla f(\bm{x}^1)^{\mathrm{T}}(\bm{x}^2 - \bm{x}^1), \; \forall \bm{x}^1, \bm{x}^2 \in X \nonumber
  \end{equation}
  が成立することである.また,$f$が強意の凸関数であるための必要十分条件は,
  \begin{equation}
    f(\bm{x}^2) > f(\bm{x}^1) + \nabla f(\bm{x}^1)^{\mathrm{T}}(\bm{x}^2 - \bm{x}^1), \; \forall \bm{x}^1, \bm{x}^2 \in X \; \mathrm{s. t.} \; \bm{x}^1 \neq \bm{x}^2 \nonumber
  \end{equation}
  が成立することである.
\end{theo}
(参考) $\bm{x} \in X$($X \subseteq \mathbb{R}^n$は開集合),$\bm{d} \in \mathbb{R}^n$とする.$f: X \rightarrow \mathbb{R}$に対して,方向微分係数(directional derivative)は,
\begin{equation}
  f^{\prime}(\bm{x}; \bm{d}) = \lim_{r \rightarrow +0} \frac{f(\bm{x} + r\bm{d}) - f(\bm{x})}{r} \nonumber
\end{equation}
と定められる.$f$が微分可能なとき,$f^{\prime}(\bm{x}; \bm{d}) = \nabla f(\bm{x})^{\mathrm{T}}\bm{d}$が成立する
\footnote{$f$が微分可能のとき,Taylorの定理より,$f(\bm{x} + r\bm{d}) = f(\bm{x}) + r \nabla f(\bm{x})^{\mathrm{T}} \bm{d} + o(r\|\bm{d}\|)$である.さらに,$\lim \limits_{r \rightarrow +0} o(r\|\bm{d}\|) / r = 0$より上式を得る.}.

\begin{theo}\label{theo:convex2}
  $X \subseteq \mathbb{R}^n$を空でない開凸集合とする.関数$f: X \rightarrow \mathbb{R}$が2階微分可能とする.
  \begin{enumerate}
    \item $f$が凸関数であるための必要十分条件は,$f$のヘッセ行列$H(\bm{x})$が任意の$\bm{x} \in X$に対して半正定(positive semi-definite),すなわち,
    \begin{equation}
      \bm{y}^{\mathrm{T}} H(\bm{x}) \bm{y} \geq 0, \; \forall \bm{y} \in \mathbb{R}^n \nonumber
    \end{equation}
    が成立することである.
    \item $f$が強意の凸関数であるための十分条件は,$f$のヘッセ行列$H(\bm{x})$が任意の$\bm{x} \in X$に対して正定(positive definite),すなわち,
    \begin{equation}
      \bm{y}^{\mathrm{T}} H(\bm{x}) \bm{y} > 0, \; \forall \bm{y} \in \mathbb{R}^n, \; \bm{y} \neq \bm{0} \nonumber
    \end{equation}
    が成立することである\footnote{この逆は一般に成立しない.}.
  \end{enumerate}
\end{theo}

\begin{description}
  \item[(凸関数の連続性)] $X \subseteq \mathbb{R}^n$を空でない凸集合とする.関数$f: X \rightarrow \mathbb{R}$が凸関数であれば,$f$は$X$の任意の内点で連続である.
  \item[Taylorの公式] $f: X \rightarrow \mathbb{R} \, (X \subseteq \mathbb{R}^n)$が$\bar{\bm{x}} \in X$の近傍で2階微分可能なとき,
  \begin{equation}
    f(\bm{x}) = f(\bar{\bm{x}}) + \nabla f(\bar{\bm{x}})^{\mathrm{T}} (\bm{x} - \bar{\bm{x}}) + \frac{1}{2} (\bm{x} - \bar{\bm{x}})^{\mathrm{T}} H((1 - \lambda)\bar{\bm{x}} + \lambda \bm{x})(\bm{x} - \bar{\bm{x}}) \nonumber
  \end{equation}
  なる$\lambda \in (0, 1)$が存在する.
\end{description}

\end{document}
