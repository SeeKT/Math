\documentclass{jsreport}
\usepackage{graphicx, url, algorithm, algorithmic, float, booktabs, listings, jlisting, color, pdfpages, amsmath, amssymb, latexsym, mathtools}
\lstset{
  basicstyle={\ttfamily},
  identifierstyle={\small},
  commentstyle={\smallitshape},
  keywordstyle={\small\bfseries},
  ndkeywordstyle={\small},
  stringstyle={\small\ttfamily},
  frame={tb},
  breaklines=true,
  columns=[l]{fullflexible},
  numbers=left,
  xrightmargin=0zw,
  xleftmargin=3zw,
  numberstyle={\scriptsize},
  stepnumber=1,
  numbersep=1zw,
  lineskip=-0.5ex
}
\usepackage[table,xcdraw]{xcolor}
\newtheorem{theo}{Theorem}[section]
\newtheorem{defi}{Definition}[section]
\newtheorem{lemm}{Lemma}[section]
\newcommand{\red}[1]{\textcolor{red}{#1}}
\newcommand{\blue}[1]{\textcolor{blue}{#1}}
\newcommand{\green}[1]{\textcolor{green}{#1}}
\renewcommand{\figurename}{図}
\renewcommand{\tablename}{表}
\renewcommand{\baselinestretch}{1.1}
\usepackage[hang,small,bf]{caption}
\usepackage[subrefformat=parens]{subcaption}
\usepackage{mathrsfs}

\begin{document}

\chapter{非線形計画問題とアルゴリズム}
この章では,非線形計画問題の最適性条件および最適化問題を解くアルゴリズムを述べる.
\section{非線形計画問題の最適性条件}
非線形計画問題においては,最適性の定義を次のように緩めることが多い.
\begin{itemize}
  \item $\bm{x}^*$は局所最小解(locally minimal solution)である.
  \begin{itemize}
    \item $\bm{x}^* \in S$に対し,ある近傍$B$が存在し,任意の$\bm{x} \in S \cap B$に対して$f(\bm{x}^*) \leq f(\bm{x})$が成立する.
  \end{itemize}
  \item $\bm{x}^*$は局所最大解(locally maximum solution)である.
  \begin{itemize}
    \item $\bm{x}^* \in S$に対し,ある近傍$B$が存在し,任意の$\bm{x} \in S \cap B$に対して$f(\bm{x}^*) \geq f(\bm{x})$が成立する.
  \end{itemize}
\end{itemize}

ここでは,最小化問題を基本としているので,局所最小解を局所最適解(locally optimal solution)ともいう.また,本来の最適解を大域最適解(globally optimal solution)という.
まず,局所最適解の必要条件,十分条件について述べ,局所最適解が大域最適解になる場合について述べる.

\subsection{制約なし問題の最適性条件}
最適化問題として,(\ref{eq:opt_nonl_nonc})を考える.
\begin{align}\label{eq:opt_nonl_nonc}
  \mathrm{minimize} &: f(\bm{x}) \nonumber\\
  \mathrm{subject \; to} &: \bm{x} \in \mathbb{R}^n
\end{align}
ただし,$f$は任意の非線形関数であり,適当な微分可能性を仮定する.

問題(\ref{eq:opt_nonl_nonc})の最適性の必要条件として,次の2つの定理を示す.
\begin{theo}\label{theo:opt_nonl_nonc_h_1}
  $f \in C^{1}$のとき,$\bar{\bm{x}} \in \mathbb{R}^n$が問題(\ref{eq:opt_nonl_nonc})の局所最適解であるための必要条件は,
  \begin{equation}
    \nabla f(\bar{\bm{x}}) = \bm{0} \nonumber
  \end{equation}
  となることである.
\end{theo}

\begin{theo}\label{theo:opt_nonl_nonc_h_2}
  $f \in C^2$のとき,$\bar{\bm{x}}$が問題(\ref{eq:opt_nonl_nonc})の局所最適解であるための必要条件は,
  \begin{equation}
    \nabla f(\bar{\bm{x}}) = \bm{0} \nonumber
  \end{equation}
  かつ,$\bar{\bm{x}}$におけるヘッセ行列$H(\bar{\bm{x}})$が半正定値,すなわち,
  \begin{equation}
    \bm{d}^{\mathrm{T}} H(\bar{\bm{x}}) \bm{d} \geq 0, \; \forall \bm{d} \in \mathbb{R}^n \nonumber
  \end{equation}
  が成立することである.
\end{theo}

さらに,局所最適解の十分条件は,定理\ref{theo:opt_nonl_nonc_j}で与えられる.
\begin{theo}\label{theo:opt_nonl_nonc_j}
  $f \in C^2$のとき,$\bar{\bm{x}}$が問題(\ref{eq:opt_nonl_nonc})の局所最適解であるための十分条件は,
  \begin{equation}
    \nabla f(\bar{\bm{x}}) = \bm{0} \nonumber
  \end{equation}
  かつヘッセ行列$H(\bar{\bm{x}})$が正定値,すなわち,
  \begin{equation}
    \bm{d}^{\mathrm{T}} H(\bar{\bm{x}}) \bm{d} > 0, \; \forall \bm{d} \in \mathbb{R}^n, \; \bm{d} \neq \bm{0} \nonumber
  \end{equation}
  が成立することである.
\end{theo}

\subsection{等式制約問題の最適性条件}
本節では,最適化問題として,(\ref{eq:opt_nonl_eq})を考える.
\begin{align}\label{eq:opt_nonl_eq}
  \mathrm{minimize} : \; &f(\bm{x}) \nonumber\\
  \mathrm{subject \; to} : \; &g_i(\bm{x}) = 0, \; i = 1, 2, \ldots, m \nonumber \\
  &\bm{x} \in \mathbb{R}^n
\end{align}

等式制約問題(\ref{eq:opt_nonl_eq})における主要な結果は,目的関数$f(\bm{x})$の代わりにLagrange関数(Lagrangian function)
\begin{equation}\label{eq:lag}
  L(\bm{x}, \bm{u}) = f(\bm{x}) + \bm{u}^{\mathrm{T}} \bm{g}(\bm{x})
\end{equation}
を用いると,前節の定理\ref{theo:opt_nonl_nonc_h_1},定理\ref{theo:opt_nonl_nonc_h_2}を自然に拡張できるという点にある.
ただし,$\bm{u} \in \mathbb{R}^m, \, \bm{g}(\bm{x}) = (g_1(\bm{x}), g_2(\bm{x}), \ldots, g_m(\bm{x}))^{\mathrm{T}}$である.

\paragraph{等式条件のペナルティ関数}
問題(\ref{eq:opt_nonl_eq})とパラメータ$k = 1, 2, \ldots$から次の関数を定義する.
\begin{equation}\label{eq:penalty}
  F^k(\bm{x}) = f(\bm{x}) + \frac{k}{2}\|\bm{g}(\bm{x})\|^2 + \frac{\alpha}{2}\|\bm{x} - \bar{\bm{x}}\|^2
\end{equation}
ただし,$\bar{\bm{x}}$は,問題(\ref{eq:opt_nonl_eq})の局所最適解である.また,$\alpha > 0$
とする.
(\ref{eq:penalty})式において,第2項は,等式制約の違反に対するペナルティと解釈することができる.第3項は,$\bm{g}(\bm{x}) = \bm{0}$を満たす解$\bm{x}$を考えるとき,$\bar{\bm{x}}$が近傍内で$F^k(\bm{x})$の唯一の局所最適解となるように導入されたものである.

$\bar{\bm{x}}$は局所最適解であるので,適当な$\varepsilon > 0$を選べば,$\bar{\bm{x}}$の近傍の閉包
\begin{equation}
  \bar{B}(\bar{\bm{x}}, \varepsilon) = \{\bm{x} \in \mathbb{R}^n \, | \, \|\bm{x} - \bar{\bm{x}}\| \leq \varepsilon\}
\end{equation}

\paragraph{等式制約問題の最適性十分条件}



\end{document}
