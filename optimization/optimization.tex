\documentclass[dvipdfmx]{jsarticle}
\input preamble.tex
\title{最適化 復習}
\author{Kosuke Toda}
\date{}
\begin{document}
\maketitle
様々な問題に対して,最も効率的になるように意思決定をする手法としてオペレーションズ・リサーチ(OR)というものがある\cite{or}.これは,現実の問題を数理モデルに置換し,問題を解決する.最適化理論(optimization theory)は,ORの基礎理論の1つであり,理論だけでなく,現実の問題を解くための手法を提供してきた\cite{opt}.本資料では,特に連続の最適化に焦点をおく.

\section{数学的基礎}

\begin{thebibliography}{99}
\bibitem{or} オペレーションズ・リサーチとは. 公益社団法人日本オペレーションズ・リサーチ学会. https://www.orsj.or.jp/whatisor/whatisor.html. (参照 2020/3/23)
\bibitem{opt} 茨木俊秀. (2011). 最適化の数学. 共立出版株式会社.
\end{thebibliography}
\end{document}
