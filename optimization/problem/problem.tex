\documentclass{jsreport}
\usepackage{graphicx, url, algorithm, algorithmic, float, booktabs, listings, color, pdfpages, amsmath, amssymb, latexsym, mathtools, ascmac, amsfonts, amsthm}
\lstset{
  basicstyle={\ttfamily},
  identifierstyle={\small},
  commentstyle={\smallitshape},
  keywordstyle={\small\bfseries},
  ndkeywordstyle={\small},
  stringstyle={\small\ttfamily},
  frame={tb},
  breaklines=true,
  columns=[l]{fullflexible},
  numbers=left,
  xrightmargin=0zw,
  xleftmargin=3zw,
  numberstyle={\scriptsize},
  stepnumber=1,
  numbersep=1zw,
  lineskip=-0.5ex
}
\usepackage[table,xcdraw]{xcolor}
\newtheorem{theo}{定理}[chapter]
\newtheorem{defi}{定義}[chapter]
\newtheorem{lemm}{補題}[chapter]
\newtheorem{prop}{命題}[chapter]
\newtheorem{coro}{系}[chapter]
\newcommand{\red}[1]{\textcolor{red}{#1}}
\newcommand{\blue}[1]{\textcolor{blue}{#1}}
\newcommand{\green}[1]{\textcolor{green}{#1}}
\renewcommand{\baselinestretch}{1.1}
\usepackage{mathrsfs}
\usepackage{bm}
\def\qed{\hfill $\Box$}
\usepackage{tikz}
\usetikzlibrary{intersections, calc, arrows}
\renewcommand\proofname{\bf 証明}

\begin{document}

\chapter{最適化問題の定式化}
本章では,一般の最適化問題について定式化を行い,用語の定義を行う.
\section{最適化問題}
最適化問題を口語的に定義すると,「与えられた条件の下で何らかの関数を最小化,もしくは最大化する問題」である.最適化問題は以下のように定義される.ここでは,最小化問題を最適化問題とする.

基礎となる空間$X$,$S \subseteq \mathbb{R}^n$,$X$上で定義された関数$f: X \rightarrow \mathbb{R}$が与えられたとき,$f$を最小にする解$\bm{x} \in X \cap X$を求める問題を最適化問題(optimization problem),あるいは計画問題(programming problem)という.つまり,最適化問題は,(\ref{eq:opt})式で定義される問題である.
\begin{align}\label{eq:opt}
  \mathrm{minimize} &: f(\bm{x}) \nonumber\\
  \mathrm{subject \; to} &: \bm{x} \in S \cap X
\end{align}
例えば$X$として,$\mathbb{R}^n$や離散集合などが考えられる.
連続最適化問題においては通常,$S$は関数$g_i: \mathbb{R}^n \rightarrow \mathbb{R}, \, i = 1, 2, \ldots, m$による不等式および等式制約を用いて,
\begin{equation}\label{eq:const}
  S = \{\bm{x} \in \mathbb{R}^n \, | \, g_i(\bm{x}) \leq 0, \, i = 1, 2, \ldots, l, \; g_i(\bm{x}) = 0, \, i = l + 1, \ldots, m\}
\end{equation}
と表現される.

\section{用語}
\begin{description}
  \item[$S \cap X$] 実行可能領域(feasible region)
  \begin{itemize}
    \item $S \cap X \neq \emptyset$ならば,実行可能(feasible)
    \item $S \cap X = \emptyset$ならば,実行不能(infeasible)
  \end{itemize}
  \item[$\bm{x} \in S \cap X$] 実行可能解(feasible solution)
\end{description}

実行可能解$\bm{x} \in S \cap X$のうち,目的関数値を最小にする解を最適解(optimal solution)という.つまり,最適化問題\ref{eq:opt}の最適解$\bm{x}^*$は,$\forall \bm{x} \in S \cap X, \, f(\bm{x}^*) \leq f(\bm{x})$を満たす$\bm{x}^* \in S \cap X$である.

最適化問題(\ref{eq:opt})において,最適解は常に存在するわけではない.実行不能であるとき,あるいはいくらでも目的関数値を小さくできる場合などでは,最適解は存在しない\footnote{実行不能であるとき,この問題を非有界(unbounded)であるといい,いくらでも目的関数値を小さくできる場合,この問題は発散する(divergent)という.}.

最適化問題は,基礎となる空間$X$,目的関数$f$,制約式(\ref{eq:const})における$g_i$により次のように分類される.
\begin{itemize}
  \item 非線形計画問題(nonlinear programming problem)
  \begin{itemize}
    \item $X = \mathbb{R}^n$
    \item $f$や$S$を定義する$g_i$に制限を置かない.
  \end{itemize}
  \item 線形計画問題(linear programming problem)
  \begin{itemize}
    \item $f, g_i$がすべて線形(1次関数).
  \end{itemize}
  \item 整数計画問題(integer programming problem)
  \begin{itemize}
    \item $X = \mathbb{Z}^n$
    \item $f, g_i$が線形.
    \item すべての変数が整数変数の全整数計画問題(all-integer programming problem),整数変数と実数変数が混在する混合整数計画問題(mixed-integer programming problem)に分類される.
  \end{itemize}
  \item 組合せ最適化問題(combinatorial optimization problem)
  \begin{itemize}
    \item 離散集合$X$に対する最適化問題.
    \item グラフ理論など.
  \end{itemize}
\end{itemize}

最適化問題(\ref{eq:opt})は,
\begin{equation}
  f^* = \inf_{\bm{x} \in S \cap X} f(\bm{x}) \nonumber
\end{equation}
を求める問題と解釈できる.$f^*$を最適値(optimal value)という.最適値$f^*$は常に定義され,最小化問題については,実行不能ならば$f^* = \infty$,発散するならば$f^* = -\infty$であり,最適解$\bm{x}^*$が存在するならば$f^* = f(\bm{x}^*)$である.

\end{document}
